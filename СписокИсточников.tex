\addcontentsline{toc}{section}{СПИСОК ИСПОЛЬЗОВАННЫХ ИСТОЧНИКОВ}

\begin{thebibliography}{9}
	\bibitem{SQL} Лукин, В.Н. Введение в проектирование баз данных. [Текст] / В.Н. Лукин. - М.: Вузовская книга, 2015. - 144 c. – ISBN 978-5-507-48747-9. – Текст : непосредственный.
    \bibitem{javascript} Фримен, А. Практикум по программированию на JavaScript / А. Фримен. – Москва~: Вильямс, 2013. – 960 с. – ISBN 978-5-8459-1799-7. – Текст~: непосредственный.
    \bibitem{PRG} Клеппман, М. Высоконагруженные приложения. Программирование, масштабирование, поддержка / М. Клеппман. – Санкт-Петербург : Питер, 2018. – 640 с. – ISBN 978-5-44-610512-0. – Текст : непосредственный.
    \bibitem{css} Веру, Л. Секреты CSS. Идеальные решения ежедневных задач / Л. Веру. – Санкт-Петербург : Питер, 2016. – 336 с. – ISBN 978-5-496-02082-4. – Текст~: непосредственный.
    \bibitem{KOD}	Мартин, Р. Чистый код. Создание, анализ и рефакторинг / Р. Мартин. – Санкт-Петербург : Питер, 2020. – 464 с. – ISBN 978-5-4461-0960-9. –
    Текст : непосредственный.
	\bibitem{ARCHMS} Баланов, А. Построение микросервисной архитектуры и разработка
	высоконагруженных приложений. Учебное пособие / А. Баланов. – Москва :
	Лань, 2024. – 244 с. – ISBN 978-5-507-48747-9. – Текст : непосредственный.
	\bibitem{htmlcss}	Дэкетт, Д. HTML и CSS. Разработка и создание веб-сайтов / Д. Дэкетт. – Москва~: Эксмо, 2014. – 480 с. – ISBN 978-5-699-64193-2. – Текст~: непосредственный.
	\bibitem{bigbook}	Макфарланд, Д. Большая книга CSS / Д. Макфарланд. – Санкт-Петербург : Питер, 2012. – 560 с. – ISBN 978-5-496-02080-0. – Текст~: непосредственный.
	\bibitem{uchiru}	Порселло, Б. React. Современные шаблоны для разработки приложений / Б. Порселло. – Санкт-Петербург : Питер, 2022. – 320 с. – ISBN 978-5-
	4461-1492-4. – Текст : непосредственный.
	\bibitem{chaynik}	Титтел, Э. HTML5 и CSS3 для чайников / Э. Титтел, К. Минник. – Москва~: Вильямс, 2016 – 400 с. – ISBN 978-1-118-65720-1. – Текст~: непосредственный.    
\end{thebibliography}
